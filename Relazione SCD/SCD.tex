\documentclass{article}

\newcommand{\NomeDocumento}{Sistemi Concorrenti e Distribuiti}
\newcommand{\Versione}{1.0}

\usepackage{fancyhdr}
\usepackage{lastpage}
\usepackage{color}
\usepackage{graphicx}
\usepackage{midpage}
\usepackage{booktabs}
\usepackage{float}
\usepackage[italian]{babel}
\usepackage{hyperref}
\hypersetup{
    colorlinks,
    citecolor=black,
    filecolor=black,
    linkcolor=black,
    urlcolor=black
}

\pagestyle{fancy}

\lfoot{\bfseries \NomeDocumento}
\cfoot{}
\rfoot{\thepage/\pageref{LastPage}}
\renewcommand{\footrulewidth}{0.4pt}

\begin{document}

\begin{titlepage} % era midpage

\begin{center}

\begin{figure}[H]
\begin{center} % \centering
\includegraphics[ width=110px]{./img/Logo_Padova.png}
\label{Logo}
\end{center} % aggiunta
\end{figure}

\begin{center}
 \vspace{1em}

 {\Large \textsc{Universit\`{a} degli Studi di Padova}}\\

\vspace{1em}

 {\Large \textsc{Corso di Laurea Magistrale in Informatica}}\\
 
 \vspace{5em}
 
 {\Large Sistemi Concorrenti e Distribuiti}\\

 \vspace{5em}

 {\LARGE \textbf{Traffico citt\`{a}}}\\

 \clearpage{\pagestyle{empty}\cleardoublepage} % puoi toglierlo, credo

\end{center}

\end{center}

\end{titlepage} % era midpage
\newpage
\null 
\newpage

\newpage
\begin{itemize}
\color{black}\item {Versione 1}
\color{green}\item {Versione 2}
\end{itemize}
\color{black}\emph{Modifiche:}
\begin{itemize}
\item {-}
\end{itemize}
\color{black}\emph{Aggiunte:}
\begin{itemize}
\item {Stesura documento}
\end{itemize}
\newpage

\tableofcontents 
\listoftables
\listoffigures
\newpage

\pagestyle{fancy}

\lhead{
	\color{black}
	\bfseries \textsf{\normalsize Introduzione\\} 
}
\chead{}
\rhead{
	\color{black}
	\bfseries \textsf{\normalsize v.\Versione\\} 
}
\section{Introduzione}
Questo documento ha lo scopo di presentare il lavoro svolto durante l\textquoteright{} attivit\`{a} di stage. 

Verr\`{a} presentata inizialmente l\textquoteright{} analisi eseguita per determinare i requisiti che dovranno essere soddisfatti dall\textquoteright{} applicazione, per poi descrivere la progettazione e la realizzazione dell\textquoteright{} applicazione.
\subsection{Problema}
- Breve descrizione \newline
- Direzioni, struttura città, ecc...\newline
- Piccola lista di requisiti e di limitazioni imposte \newline 
\newpage

\pagestyle{fancy}

\lhead{
	\color{black}
	\bfseries \textsf{\normalsize Esecuzione\\} 
}
\chead{}
\rhead{
	\color{black}
	\bfseries \textsf{\normalsize v.\Versione\\} 
}

\section{Esecuzione}
\subsection{SBT}
La piattaforma di Android offre due importantissime componenti per la realizzazione di applicazioni, ovvero le activity ed i service.
\subsection{Scala}
Un\textquoteright{} activity \`{e} una componente dell\textquoteright{} applicazione provvista di una interfaccia grafica. Sono quindi usate principalmente per operazioni che richiedono una interazione con l\textquoteright{} utente.

\subsection{Akka}
Un\textquoteright{} activity \`{e} una componente dell\textquoteright{} applicazione provvista di una interfaccia grafica. Sono quindi usate principalmente per operazioni che richiedono una interazione con l\textquoteright{} utente.

\newpage

\pagestyle{fancy}

\lhead{
	\color{black}
	\bfseries \textsf{\normalsize Studio del problema\\} 
}
\chead{}
\rhead{
	\color{black}
	\bfseries \textsf{\normalsize v.\Versione\\} 
}

\section{Studio del problema}
- Gerarchia \newline
- Motivazione scelta gerarchia
\subsection{Entit\`{a}}
- Lista entità
\subsection{Stati}
- Lista stati
\subsection{Azioni}
- Tabellona
\subsection{Sincronizzazione}
- Non so tuttora se sia giusto

\newpage

\pagestyle{fancy}

\lhead{
	\color{black}
	\bfseries \textsf{\normalsize Studio del problema\\} 
}
\chead{}
\rhead{
	\color{black}
	\bfseries \textsf{\normalsize v.\Versione\\} 
}

\section{Implementazione}
\subsection{Progettazione}
- Immagine schema
\subsubsection{Linguaggio}
Il linguaggio scelto per realizzare il progetto \`{e} Scala. Tramite l\textquoteright{} uso del toolkit Akka, \`{e} infatti possibile applicazioni concorrenti, distribuite e resistenti agli errori. Il modello ad attori permette di gestire in maniera più semplice ed efficiente i problemi di concorrenza.

\`{E} permessa inoltre una buona gestione degli errori. Ogni attore \`{e} supervisionato da un altro attore. Ogni attore che provocher\`{a} un errore dovr\`{a} infatti riferire il problema al suo supervisore, il quale dovr\`{a} reagire in maniera appropriata.
Akka permette anche di realizzare applicazioni distribuite. \`{E} possibile infatti creare attori in diversi nodi all\textquoteright{} interno di un cluster. Tali attori potranno comunicare senza alcun problema nonostante vengano eseguiti in nodi diversi.
\subsubsection{Concorrenza}
- Assenza di lock, condisioni e cagate varie \newline
- Semafori e priorità
\subsubsection{Distribuzione}
- Sincronizzazioni iniziali \newline
- Caduta zone \newline
- Invio messaggi e ACK
\subsubsection{Lista classi}
- Lista classi, breve descrizione e funzionamento
\subsubsection{Descrizione classi}
- Diagramma delle classi
\subsubsection{Correttezza ordinamento messaggi}
- Scrivo che ogni messaggio arriva prima di altri (aggiorno con la sincronizzazione iniziale)
\subsection{Funzionalit\`{a} non implementate}
- Motivazioni \newline
- Salvataggio di stato ogni tanto (quando una zona torna su, dovrebbe avere un metodo di recupero dei mezzi che aveva) \newline
- Grafica (creazione, ogni corsia, marciapiede, tratto e striscia pedonale ha un riferimento ad una label che modificherebbe man mano) \newline
\subsection{Screenshots}
- Fermata autobus con pedoni che salgono e scendono \newline
- Code ai semafori \newline
- Caduta zone e deviazioni

\newpage

\pagestyle{fancy}

\lhead{
	\color{black}
	\bfseries \textsf{\normalsize Bibliografia\\} 
}
\chead{}
\rhead{
	\color{black}
	\bfseries \textsf{\normalsize v.\Versione\\} 
}
\section{Bibliografia}
\begin{itemize}
\item {[1]} Scala: \textit{http://www.scala-lang.org/documentation/}
\item {[2]} Akka: \textit{http://doc.akka.io/docs/akka/snapshot/scala.html}
\item {[3]} SBT: \textit{http://www.scala-sbt.org/}
\item {[4]} Sistemi Concorrenti e Distribuiti: \textit{http://www.math.unipd.it/$\sim$tullio/SCD/2014/}
\item {[5]} Routing: \textit{http://en.wikipedia.org/wiki/List\_of\_ad\_hoc\_routing\_protocols}
\end{itemize}
\begin{midpage}
\begin{center}
\null
\end{center}
\end{midpage}
\end{document}
